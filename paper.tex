\documentclass{article}
\usepackage[utf8]{inputenc}
\usepackage[margin=1.2in]{geometry}

\title{BCI project}
\author{Wouter Eijlander, Margo van der Stam, \\Wouter van der Weel, Koen Dercksen, Chris Kamphuis}
\date{November 2014}

\begin{document}

\maketitle
\begin{abstract}
Abstract stub.
\end{abstract}

\section{Introduction}
Inspired by P. Sauseng et al.\cite{Sauseng2005} we want to do research on the subject of $\alpha$-wave power in the visual cortex when covertly attending to stimuli. 

\section{Hypothesis}
We expect that the power of the $\alpha$-waves is lower on the ipsilateral side of the brain in the visual cortex (V1) when subjects are covertly attending to a stimulus.

\section{Experiment}
% We present our subjects a slideshow in which they have to covertly attend to a feature on the right or left side of the presented slide. For each subject the feature is always on the same side of the slide in order to ensure that there are no attentional shifts. Subjects will be instructed to report as many of the features as possible afterwards. Because of this, they have to remember as much as possible of these features. While they are looking at the slideshow, we measure the brain activity in terms of power of the $\alpha$-waves in the V1 areas of their brain. The measuring is done with an EEG device.

We present two slidesets per subject; one that gives target cues on the right side of the focus point and one that gives target cues on the left side of the focus point; this is to avoid attentional shifts. Subjects are asked to covertly attend to the cues and count the number of target cues presented. Target cues are green squares. The task is to count the number of green squares presented.

During the presentation of these cues, we measure the brain activity in terms of the power of the $\alpha$-waves in the V1 areas of their brain using EEG.

\section{Results}

\section{Discussion}

\bibliographystyle{unsrt}
\bibliography{references}

\end{document}