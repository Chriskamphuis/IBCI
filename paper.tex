\documentclass{article}
\usepackage[utf8]{inputenc}
\usepackage[margin=1.2in]{geometry}

\title{BCI project}
\author{Wouter Eijlander, Margo van der Stam, 
				\\Wouter van der Weel, Koen Dercksen, Chris Kamphuis}
\date{November 2014}

\begin{document}

\maketitle
\begin{abstract}
	Inspired by P. Sauseng et al.\cite{Sauseng2005} we want to doe research on
	the subject of $\alpha$-wave power in the visual cortex when covertly
	attending to stimuli.
\end{abstract}

\section{Introduction}
\subsection{$\alpha$-waves}
The $\alpha$-rhythm is an neural oscillation in the range of 8 - 12 Hz. 
$\alpha$-waves are the strongest in subjects who rest and have their eyes
closed. In humans the $\alpha$-waves are produced in the region of the visual
cortex(V1) of the brain. 
\subsection{Covert Attention}
Covertly attending means that a subject is attending to something in the
periphery of his visual field. This can be done by shifting your attention to
an object which is not perceived by the macula lutea.
\subsection{Lateralization}
According to the literature *CITE*, the power of the $\alpha$-waves on the
contralateral side of your vision in the V1 is lower than the power on the
ipsilateral side. This due to the fact that the neurons on the ipsilateral side
fire simultaneously in order to inhibit the input. This way the input on the
contralateral side gets through, so you can selectively attend to a part in
your visual field. % Perhaps need to be more scientific.
\subsection{Electroencephalography}
An electroencephalography (EEG) is a recording of electrical activity on the
outside of the head. The brain activity has to be strong to be measured with an
EEG since it records a lot of noise and the skull destoys a lot of the signal.
As opposed to the poor spatial resolution, an EEG has a great temporal
resolution. Moreover, an EEG has low costs in comparison to other brain
activity recorders. 
\subsection{How to Analyze}
Since there is a lot of noise in EEG recordings, you can not analyze single
recordings of EEG's. Therefor you want to use multiple recordings and average
the noise out. However, since $\alpha$-power is not time-locked, you have to
average in the frequency domain. If you then find variations in the power of
the $\alpha$ you can analyze it.


\section{Hypothesis} 
We expect that the power of the $\alpha$-waves is lower on
the ipsilateral side of the brain in the visual cortex (V1) when subjects are
covertly attending to a stimulus.

\section{Experiment} 
We present two slidesets per subject; one that presents
target cues on the right side of the focus point and one that presents target
cues on the left side of the focus point. Both slide sets contain an initial
stimulus consisting of a fixation cross and two green squares presented on the
left and right side. Both squares are presented in the lower half of the visual
field, as Royendijk et al. (2013) have found to be the most efficient. After
the initial stimulus has been presented for a random duration between 0.5 and 3
seconds, a slight change in the square at the cued direction is presented for
0.05 seconds. 
The possible changes are: 
\begin{itemize} 
	\item A small shift downwards.  
	\item A small shift outwards.  
	\item A small increase in size.
  \item A small change in color.  
\end{itemize} In order to avoid attentional
	shifts and to either side as well as attentional rythms we randomized the cue
	direction for each epoch as well as the duration of the initial stimulus.
	Stimuli on both sides are presented at 7\degree eccentricity from the
	direction of the eyes, as Royendijk et al. (2013) have found to be sufficient
	for 'difficult cues'.  Subjects are asked to covertly attend to the cues and
	count the number of target cues presented. Target cues are a specific change
	in the nature of the green square, namely colour change. 

During the presentation of these cues, we measure the brain activity in terms
of the power of the $\alpha$-waves in the V1 areas of their brain using EEG.

\section{Results}

\section{Conclusion}
As expected our hypotesis was true. When you are are covertly attending to a
stimulus, the power of the $\alpha$-waves on the contralateral side of the V1
are lower than the power on the ipsilateral side of the V1.

\section{Discussion}
\subsection{Trials}
In order to find better results we should have tested on more subjects and we
had run more trials on each subjects. However, do to limits in time we couldn't
be we are sure that we, if given more time, could find better results.
\subsection{Hardware}
We used a mobita EEG recorder for our experiment, there was however another EEG
recorder available which probally would have given us better results. However
due to the time it cost setting that recorder up we chose for the mobita. We
however argue that for this experiment the mobita was good enough.

\bibliographystyle{unsrt} \bibliography{references}

\end{document}
